\documentclass[utf8]{ctexart}
\usepackage{geometry,amsmath,amssymb,theorem,caption,extarrows,mathrsfs,physics,bm}
\usepackage{graphicx,xcolor,listings,geometry,booktabs,subfigure,tikz}
\usepackage{pgfplots,grffile}
\pgfplotsset{compat=newest}
  %% the following commands are needed for some matlab2tikz features
\usetikzlibrary{plotmarks}
\usetikzlibrary{arrows.meta}
\usetikzlibrary{calc}
\usepgfplotslibrary{patchplots}
\usepackage{xeCJK,fontspec}
\setCJKmainfont[BoldFont=LXGW WenKai Bold]{LXGW WenKai Mono}
  \setCJKsansfont{黑体}%serif是有衬线字体sans serif无衬线字体。
  %\setmonofont{CMU Typewriter Text} % 等寬字型
  \XeTeXlinebreaklocale "zh"
  \XeTeXlinebreakskip = 0pt plus 1pt minus 0.1pt
\lstset{
    basicstyle          =   \sffamily,          % 基本代码风格
    keywordstyle        =   \bfseries,          % 关键字风格
    commentstyle        =   \rmfamily\itshape,  % 注释的风格,斜体
    stringstyle         =   \ttfamily,  % 字符串风格
    flexiblecolumns,                % 别问为什么,加上这个
    numbers             =   left,   % 行号的位置在左边
    showspaces          =   false,  % 是否显示空格,显示了有点乱,所以不现实了
    numberstyle         =   \zihao{-5}\ttfamily,    % 行号的样式,小五号,tt等宽字体
    showstringspaces    =   false,
    captionpos          =   t,      % 这段代码的名字所呈现的位置,t指的是top上面
    frame               =   tb,   % 显示边框
}

\lstdefinestyle{Python}{
    language        =   Python, % 语言选Python
    basicstyle      =   \zihao{-5}\ttfamily,
    numberstyle     =   \zihao{-5}\ttfamily,
    keywordstyle    =   \color{blue},
    keywordstyle    =   [2] \color{teal},
    stringstyle     =   \color{magenta},
    commentstyle    =   \color[HTML]{338AAF}\ttfamily,
    breaklines      =   true,   % 自动换行,建议不要写太长的行
    columns         =   fixed,  % 如果不加这一句,字间距就不固定,很丑,必须加
    basewidth       =   0.5em,
}
\definecolor{codegreen}{rgb}{0,0.6,0}
\definecolor{codegray}{rgb}{0.5,0.5,0.5}
\definecolor{codepurple}{rgb}{0.58,0,0.82}
\definecolor{backcolour}{rgb}{0.95,0.95,0.92}
\definecolor{hlinkblue}{rgb}{0.27,0.52,0.76}
\lstdefinestyle{mathematica}{
    backgroundcolor=\color{backcolour},
    commentstyle=\color[HTML]{338AAF}\ttfamily,
    keywordstyle=\zihao{-5}\sffamily\bfseries\color{magenta},
    numberstyle=\tiny\color{codegray},
    stringstyle=\color{codepurple},
    basicstyle=\zihao{-5}\ttfamily,
    breakatwhitespace=false,
    breaklines=true,
    basewidth=0.5em,
    captionpos=b,
    columns=fixed,
    keepspaces=true,
    numbers=left,
    numbersep=5pt,
    showspaces=false,
    showstringspaces=false,
    showtabs=false,
    tabsize=4
}
\lstdefinestyle{matlab}{
    language=matlab,
    backgroundcolor=\color{backcolour},
    commentstyle=\color[HTML]{338AAF}\ttfamily,
    keywordstyle=\zihao{-5}\sffamily\bfseries\color{magenta},
    numberstyle=\tiny\color{codegray},
    stringstyle=\color{codepurple},
    basicstyle=\zihao{-5}\ttfamily,
    breakatwhitespace=false,
    breaklines=true,
    basewidth=0.5em,
    captionpos=b,
    columns=fixed,
    keepspaces=true,
    numbers=left,
    numbersep=5pt,
    showspaces=false,
    showstringspaces=false,
    showtabs=false,
    tabsize=4
}
\newcommand{\dif}{\mathop{}\!\mathrm{d}}
\newcommand{\const}{\mathop{}\!\mathrm{const.}}
\newcommand{\splitline}{\noindent\rule[0.25\baselineskip]{\textwidth}{0.5pt}}
\newcommand{\autographinsert}[2]{\includegraphics[
  height=\dimexpr\pagegoal-\pagetotal-4\baselineskip\relax,width=#1\textwidth,
  keepaspectratio]{#2}}
  % NOTE: 插入图片如果出问题飘到下一页,请调整减掉的行数
\newtheorem{theorem}{定理}
\geometry{a4paper,left=1.8cm,right=1.8cm,top=1.5cm,bottom=1.5cm}
\begin{document}
\title{Quiz 3b }
\author{仇琨元}
\date{\today}
\maketitle

\section{Problem 1}

\textbf{Sol:}  True.

Since \(f(t,y)=y^{2}\) is continuous and differentiable on \(\mathbb{R}\), apply the existence and uniqueness theorem over the region

\begin{equation}
	\Omega=(-a,a)\times (-b,b)
\end{equation}
beneath the desired initial value \(y(0)=0\):

\begin{equation}
	\begin{aligned}
		\left.        \left|f(t,y)\right|\right|_{(t,y)=(0,0)} & \leq b^{2} \\
		\Rightarrow M                                          & =b^{2}
	\end{aligned}
\end{equation}

Therefore, a unique solution exists at the desired point \((t,y)=(0,0)\) over the interval \([-h,h],h=\left|\min \{a,b\}\right|\).

\section{Problem 2}

\textbf{Sol}:

Use the existence and uniqueness theorem over the region

\begin{equation}
	\Omega=(-2,2)\times (-b,b)
\end{equation}
including the initial value \(y(0)=0\):

\begin{equation}
	\begin{aligned}
		\left.        \left|f(t,y)\right|\right|_{(0,0)} & \leq\left|b\right|^{\frac{1}{3}} \\
		\Rightarrow M                                    & =\left|b\right|b^{\frac{1}{3}}   \\
		h                                                & =\min \{2,b^{\frac{2}{3}}\}
	\end{aligned}
\end{equation}

Therefore, the set of values that the solutions satisfying the IVP can have contains only one single element \(\{(8\sqrt{3})/9\}\)since the uniqueness is present.

\begin{equation}
	\begin{aligned}
		y'(t)                               & =y^{\frac{1}{3}} \\
		\Rightarrow y^{- \frac{1}{3}}\dif y & =\dif t          \\
		\frac{3}{2}y^{\frac{2}{3}}          & =t+C_1           \\
		y(0)=0\Rightarrow C_1               & =0
	\end{aligned}
\end{equation}

\section{Problem 3}

\textbf{Sol:}

(i)

If the \((x,y)\) satisfies \(\sin x =0\), the equation degenerates to

\begin{equation}
	n\sin y=0
\end{equation}

Therefore, \((x,y)=\pi(k_1,k_2),\{k_1,k_2\}\subset \mathbb{Z}\) is a solution.

(ii)

If the \((x,y)\) satisfies \(\sin y=0\), the equation degenerates to

\begin{equation}
	\begin{aligned}
		\sin x \frac{\mathrm{d}y}{\mathrm{d}x}           & =0         \\
		\int \frac{1}{\sin x}\dif x                      & =y+C_1     \\
		                                                 & =k \pi+C_1 \\
		\Rightarrow \log \left(\sin \left(\frac{x}{2}\right)\right)
		-\log \left(\cos \left(\frac{x}{2}\right)\right) & =k\pi+C_1
	\end{aligned}
\end{equation}

This equation gives out all the possible \(x\) regarding some initial conditions.

(iii)

If \(\sin x \sin y \not =0\), the equation is separable.

\begin{equation}
    \begin{aligned}
        \cot y \dif y&=-n \cot x \dif x\\
        \Rightarrow \ln (\sin (y))&=-n \ln (\sin (x))+C_1\\
        y&=\pm \cos ^{-1}\left(\frac{1}{2} c_1 \sec^{n} (x)\right)
    \end{aligned}
\end{equation}

\section{Problem 4}

\textbf{Sol:}

Use the Euler's method

\begin{equation}
    \begin{cases}
        y[n+1]&=y[n]+f(t[n],y[n])\delta \\
        t[n+1]&=t[n]+\delta
    \end{cases}
\end{equation}
with the provided equation

\begin{equation}
    y'(t)=y(3-ty),y(0)=0.5,\delta=0.1
\end{equation}

Use the MATLAB to obtain the solution sequence:
\begin{lstlisting}[language=matlab,style=matlab]
x=[0,0.1,0.2,0.3,0.4,0.5];
y=[0.5,0.65,0.8407,1.078,1.368,1.703]
\end{lstlisting}

\end{document}