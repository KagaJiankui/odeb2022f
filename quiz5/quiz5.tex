\documentclass[utf8]{ctexart}
\usepackage{geometry,amsmath,amssymb,theorem,caption,extarrows,mathrsfs,physics,bm}
\usepackage{graphicx,xcolor,listings,geometry,booktabs,subfigure,tikz}
\usepackage{pgfplots,grffile}
\pgfplotsset{compat=newest}
  %% the following commands are needed for some matlab2tikz features
\usetikzlibrary{plotmarks}
\usetikzlibrary{arrows.meta}
\usetikzlibrary{calc}
\usepgfplotslibrary{patchplots}
\usepackage{xeCJK,fontspec}
\setCJKmainfont[BoldFont=LXGW WenKai Bold]{LXGW WenKai Mono}
\setCJKsansfont{黑体}
  \XeTeXlinebreaklocale "zh"
  \XeTeXlinebreakskip = 0pt plus 1pt minus 0.1pt
\lstset{
    basicstyle          =   \sffamily,          % 基本代码风格
    keywordstyle        =   \bfseries,          % 关键字风格
    commentstyle        =   \rmfamily\itshape,  % 注释的风格,斜体
    stringstyle         =   \ttfamily,  % 字符串风格
    flexiblecolumns,                % 别问为什么,加上这个
    numbers             =   left,   % 行号的位置在左边
    showspaces          =   false,  % 是否显示空格,显示了有点乱,所以不现实了
    numberstyle         =   \zihao{-5}\ttfamily,    % 行号的样式,小五号,tt等宽字体
    showstringspaces    =   false,
    captionpos          =   t,      % 这段代码的名字所呈现的位置,t指的是top上面
    frame               =   tb,   % 显示边框
}

\lstdefinestyle{Python}{
    language        =   Python, % 语言选Python
    basicstyle      =   \zihao{-5}\ttfamily,
    numberstyle     =   \zihao{-5}\ttfamily,
    keywordstyle    =   \color{blue},
    keywordstyle    =   [2] \color{teal},
    stringstyle     =   \color{magenta},
    commentstyle    =   \color[HTML]{338AAF}\ttfamily,
    breaklines      =   true,   % 自动换行,建议不要写太长的行
    columns         =   fixed,  % 如果不加这一句,字间距就不固定,很丑,必须加
    basewidth       =   0.5em,
}
\definecolor{codegreen}{rgb}{0,0.6,0}
\definecolor{codegray}{rgb}{0.5,0.5,0.5}
\definecolor{codepurple}{rgb}{0.58,0,0.82}
\definecolor{backcolour}{rgb}{0.95,0.95,0.92}
\definecolor{hlinkblue}{rgb}{0.27,0.52,0.76}
\lstdefinestyle{mathematica}{
    backgroundcolor=\color{backcolour},
    commentstyle=\color[HTML]{338AAF}\ttfamily,
    keywordstyle=\zihao{-5}\sffamily\bfseries\color{magenta},
    numberstyle=\tiny\color{codegray},
    stringstyle=\color{codepurple},
    basicstyle=\zihao{-5}\ttfamily,
    breakatwhitespace=false,
    breaklines=true,
    basewidth=0.5em,
    captionpos=b,
    columns=fixed,
    keepspaces=true,
    numbers=left,
    numbersep=5pt,
    showspaces=false,
    showstringspaces=false,
    showtabs=false,
    tabsize=4
}
\lstdefinestyle{matlab}{
    language=matlab,
    backgroundcolor=\color{backcolour},
    commentstyle=\color[HTML]{338AAF}\ttfamily,
    keywordstyle=\zihao{-5}\sffamily\bfseries\color{magenta},
    numberstyle=\tiny\color{codegray},
    stringstyle=\color{codepurple},
    basicstyle=\zihao{-5}\ttfamily,
    breakatwhitespace=false,
    breaklines=true,
    basewidth=0.5em,
    captionpos=b,
    columns=fixed,
    keepspaces=true,
    numbers=left,
    numbersep=5pt,
    showspaces=false,
    showstringspaces=false,
    showtabs=false,
    tabsize=4
}
\newcommand{\dif}{\mathop{}\!\mathrm{d}}
\newcommand{\const}{\mathop{}\!\mathrm{const.}}
\newcommand{\splitline}{\noindent\rule[0.25\baselineskip]{\textwidth}{0.5pt}}
\newcommand{\autographinsert}[2]{\includegraphics[
  height=\dimexpr\pagegoal-\pagetotal-4\baselineskip\relax,width=#1\textwidth,
  keepaspectratio]{#2}}
  % NOTE: 插入图片如果出问题飘到下一页,请调整减掉的行数
\newtheorem{theorem}{定理}
\geometry{a4paper,left=1.8cm,right=1.8cm,top=1.5cm,bottom=1.5cm}
\begin{document}
\title{ODE B Quiz 5}
\author{仇琨元}
\date{\today}
\maketitle

\section{Problem 1}

Use the exponential intrinsic function \(e^{\omega t}\).
\begin{equation}
	\begin{aligned}
		9\ddot{y} +9\dot{y}-4y              & =0                                     \\
		e^{\omega t}(9\omega^{2}+9\omega-4) & =0                                     \\
		\Rightarrow \omega_1,\omega_2       & =\left(\frac{1}{3},-\frac{4}{3}\right)
	\end{aligned}
	\label{eq1-prob1}
\end{equation}

Therefore, the general solution of the differential equation is
\begin{equation}
	y(t)=A \mathrm{e}^{\frac{t}{3}}+B \mathrm{e}^{-\frac{4}{3}t}
	\label{eq2-prob1}
\end{equation}

\section{Problem 2}

Apply the order reduction on the given equation. Take
\begin{equation}
	y_2=u(t)y_1 (t)=u(t)t^{-\frac{1}{2}}\sin (t)
	\label{eq1-prob2}
\end{equation}

Then the equation can be rearranged into the standard 2nd order form \(\ddot{y}+p(t)\dot{y}+q(t)y=0\):

\begin{equation}
	\begin{aligned}
		p(t)                                        & =t^{-1}                                                               \\
		q(t)                                        & =1-0.25t^{-2}                                                         \\
		K(t)                                        & =\dot{c}=t^{-\frac{1}{2}}\cos (t)-\frac{1}{2}t^{-\frac{3}{2}}\sin (t) \\
		\Rightarrow 2K(t)u'(t)+(p(t)K(t)+K'(t))u(t) & =0
	\end{aligned}
	\label{eq2-prob2}
\end{equation}

Solve the equation \ref{eq2-prob2}, the coefficient \(u(t)\) and the another linearly independent specified solution \(y_2(t)=u(t)y_1(t)\) can be obtained:
\begin{equation}
	\begin{aligned}
		u(t)   & =C_1 \exp\left(\frac{\ln (t)-2 \ln (2 t \cos (t)-\sin (t))}{4}\right)                          \\
		y_2(t) & =C_1 t^{-\frac{1}{2}}\sin (t) \exp\left(\frac{\ln (t)-2 \ln (2 t \cos (t)-\sin (t))}{4}\right)
	\end{aligned}
\end{equation}

\section{Problem 3}

Solve the non-homogeneous equation with Laplace transform.
\begin{equation}
	\begin{aligned}
		(1+s^{2})Y(s)    & =\frac{12s}{(4+s^{2})^{2}}+\frac{s}{4+s^{2}} \\
		                 & =\frac{s}{4+s^{2}} \frac{16+s^{2}}{4+s^{2}}  \\
		\Rightarrow Y(s) & =\frac{s(16+s^{2})}{(4+s^{2})(1+s^{2})}
	\end{aligned}
\end{equation}

Inverse Laplace transform gives out the general solution:
\begin{equation}
	y(t)=\sin (t) y'(0)+\cos (t) \left(-2 t \sin (t)+y(0)+\frac{5}{3}\right)-\frac{5}{3}\cos (2 t)
\end{equation}

\end{document}